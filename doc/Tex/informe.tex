\documentclass[a4paper,10pt]{article}
%\documentclass[a4paper,10pt]{scrartcl}

\usepackage[utf8]{inputenc}

\title{$C^3$:Cloud Code Campus}
\author{Bogado Sebastian 	91707\\
García Marra Alejandro 	91516\\
Romero Hector 		92248}
\date{}

\begin{document}
\maketitle

 
	%CCC: Cloud Code Campus (C³ )

%Integrantes:
%	Bogado Sebastian 		91707
%	García Marra Alejandro 		91516
%	Romero Hector 			92248 

\section{Objetivo}

Desarrollar una plataforma web para aprendizaje de programación, resolución de ejercicios y desarrollo de código de forma comunitaria, la cual puede tomar la forma de clase online, grupo de estudio o grupo de resolución de problemas abierto (Comunidad).

Para cualquier Group se tiene un Admin y puede tener varios SuperUser que lo administren. Estos controlan todo lo relacionado con los Members de dicho Group, los Projects y sus Solutions subidas. 
Cada Project puede tener n Solutions asociadas, cada una con un puntaje, y siempre se tiene una "Best Solution". 

Cada Solution tiene asociada una sección de Comments, un tracking de Fixes y versiones. El Solution Edit está relacionado con los permisos que decida habilitar el Admin. Además, se da soporte para descarga de los archivos que componen la Solution.

\subsection{Groups}

Un Group puede ser:
\begin{itemize}
 	\item Open:   Cualquiera puede hacerse Member y subir Solutions. 
		Cualquiera puede proponer Projects para que el Admin o SuperUsers lo aprueben
		Un Guest de la web puede verlo y ver sus Projects y Solutions.

	\item Closed: Para hacerse Member hay que hacer un Request al Admin o SuperUsers.
		Idem Open.
		Sólo los Users pueden verlo, asi como también sus Projects y Solutions.

	\item Private: Puede hacerse Member sólo por Invite de un Member existente.
		 Idem Closed.
		 Sólo sus Members pueden verlo.

	\item Class:  Tipo especial de Private, donde:
			Admin = Professor
			SuperUser = Assistants or Collaborators
			Members = Students
		Sólo pueden enviar Invites el Professor o los Assistants.
		Solo los Professors y Assistants pueden subir Projects.
\end{itemize}

\subsection{Users}

Se dice que todos los Users forman la Community.

Cualquier User puede, por su cuenta, abrir un "Community Project", donde todo User puede participar con su Solution.

\subsection{Class and Grades}

En el formato Class, Professor y SU (a partir de ahora "admins"), junto con un Project pueden subir los lineamientos básicos de resolución. Esto puede ser en la forma de links, textos o archivos.
También en este formato, los admins plantean distintos Deadlines asociados a cada Project. El primero es para la subida de Solutions, el segundo para la etapa de "Peer Reviews" y el tercero para la etapa de "Comments Only".

Durante la primer etapa, las Solutions son privadas para los Students y solo pude verla el Owner y los admins. 
Durante el Peer Review se bloquea la subida de solutions y se lleva a cabo la calificación de trabajos entre pares.

Las calificaciones o Grades de una Class, son análogas a los Scores de las Solutions en el resto de los Groups.
Todo Grade o Score está ponderado por distintos factores como: Average Score (AS) de sus Solutions, sus participaciones (Comments con buen AS).
Las Grades, ademas, se ven afectadas por el tiempo de subida de la Solution y los Grades asignados por los admins.
El Admin de cualquier group puede habilitar distintos Scores, por ejemplo, Code Quality, Solution Effectiveness, Solution Efficienci, etc.
Si el Admin de cualquier Group así lo desea, los Scores pueden ser Private o Public. Public puede ser únicamente el número o número y nombre del User que la asignó.

En una clase, se pueden establecer trabajos iterativos de mejora sobre soluciones previas. Esto puede ser "mejorar la solución propia subida antes para que haga X" o "trabajar sobre la mejor solución del Project previo para lograr X", etc.

Una clase puede tener un Class Deadline, por ejemplo un cuatrimestre, a partir del cual se termina la asociación de Students y admins. Los admins pueden elegir si los Projects quedan en forma Public, Private, Historical (Solo puede leerse, sin comentar) o se eliminan.

Cada User se queda con una copia propia asociada a su cuenta de cada una de sus Solutions presentadas. 

Los Scores adquiridos en Class o Private Group no se transfieren para el promedio a los Community, Closed o Public Groups.

\end{document}
